 % arara: pdflatex: { synctex: yes }
% arara: makeindex: { style: ctuthesis }
% arara: bibtex

% The class takes all the key=value arguments that \ctusetup does,
% and a couple more: draft and oneside
\documentclass[twoside]{ctuthesis}
\usepackage{graphicx}
\usepackage[sortlocale=cs_CZ, sorting=debug, style=iso-numeric, maxnames=2]{biblatex}
\usepackage{array,etoolbox}
\preto\tabular{\setcounter{magicrownumbers}{0}}
\newcounter{magicrownumbers}
\newcommand\rownumber{\stepcounter{magicrownumbers}\arabic{magicrownumbers}}

\usepackage{indentfirst}
\usepackage{multirow}

\graphicspath{{img/}}
\renewcommand*{\finalnamedelim}{ a }
\ctusetup{
	mainlanguage = czech,
	otherlanguages = {slovak, english},
	title-czech = {Mobilní aplikace pro plánování lidských zdrojů},
	title-english = {TODO},
	doctype = S,
	faculty = F3,
	department-czech = {Katedra počítačů},
	department-english = {Department of Computer Science},
	author = {Martina Kopecká},
	supervisor = {Prof. Krutoš Spravedlivý},
	supervisor-address = {TODO},
	fieldofstudy-english = {Software Engineering -- TODO, co je ENG nazev},
	fieldofstudy-czech = {Softwarové inženýrství a technologie},
	keywords-czech = {slovo, klíč},
	keywords-english = {word, key},
	day = 1,
	month = 1,
	year = 2021,
	pkg-listings = true,
}

\ctuprocess

\addto\ctucaptionsczech{%
	\def\supervisorname{Vedoucí}%
	\def\subfieldofstudyname{Studijní program}%
}

\ctutemplateset{maketitle twocolumn default}{
	\begin{twocolumnfrontmatterpage}
		\ctutemplate{twocolumn.thanks}
		\ctutemplate{twocolumn.declaration}
		\ctutemplate{twocolumn.abstract.in.titlelanguage}
		\ctutemplate{twocolumn.abstract.in.secondlanguage}
		\ctutemplate{twocolumn.tableofcontents}
		\ctutemplate{twocolumn.listoffigures}
	\end{twocolumnfrontmatterpage}
}

\definecolor{codegreen}{rgb}{0,0.6,0}
\definecolor{codegray}{rgb}{0.5,0.5,0.5}
\definecolor{codepurple}{rgb}{0.58,0,0.82}
\definecolor{backcolour}{rgb}{0.88,0.96,0.98}

\lstdefinestyle{mystyle}{
    backgroundcolor=\color{backcolour},
    commentstyle=\color{codegreen},
    keywordstyle=\color{magenta},
    numberstyle=\tiny\color{codegray},
    stringstyle=\color{codepurple},
    basicstyle=\ttfamily\footnotesize,
    breakatwhitespace=false,
    breaklines=true,
    captionpos=b,
    keepspaces=true,
    numbers=left,
    numbersep=5pt,
    showspaces=false,
    showstringspaces=false,
    showtabs=false,
    tabsize=2,
		extendedchars=false,
    inputencoding=utf8,
		texcl=true,
		literate={é}{{\'e}}1
           {č}{{\v{c}}}1
           {ľ}{{\v{l}}}1
           {ť}{{\v{t}}}1
           {ý}{{\'y}}1
           {ě}{{\v{e}}}1
           {ř}{{\v{r}}}1
           {š}{{\v{s}}}1
           {ž}{{\v{z}}}1
           {á}{{\'a}}1
           {í}{{\'i}}1
           {ó}{{\'o}}1
           {ň}{{\v{n}}}1
           {ď}{{\v{d}}}1
           {ú}{{\'u}}1
           {ů}{{\r{u}}}1
           {ĺ}{{\v{l}}}1
}


\setlength{\parskip}{0.5em}
% \setlength{\parskip}{5ex plus 0.2ex minus 0.2ex}

% Abstract in Czech
\begin{abstract-czech}
Český abstrakt
\end{abstract-czech}

% Abstract in English
\begin{abstract-english}
English abstract
\end{abstract-english}

% Acknowledgements / Podekovani
\begin{thanks}
Děkuji ČVUT, že mi je tak dobrou \emph{alma mater}.
\end{thanks}

% Declaration / Prohlaseni
\begin{declaration}
Prohlašuji, že jsem předloženou práci vypracovala samostatně s~použitím uvedené literatury.

V Praze, \ctufield{day}.~\monthinlanguage{title}~\ctufield{year}
\end{declaration}

\DeclareLabeldate[article]{
  \field{date}
  \field{year}
  \field{eventdate}
  \field{origdate}
  \field{urldate}
}

\addbibresource{bibliography.bib}
\lstset{style=mystyle}

\begin{document}

% \maketitle
%
% \chapter{Úvod}
% TODO


\chapter{Analýza problematiky}
Následující kapitola bude věnována analýze problematiky plánování zdrojů, a to jednak z~hlediska terminologie, jednak z~hlediska manažerského a právního (důraz je zde kladen především na pracovní dobu a typy pracovněprávních vztahů v~české legislativě a rozdíly mezi nimi).

\section{Definice a terminologie}

Řízení lidských zdrojů (angl. human resource management\footnote{Také se užívá pojem people management, neboť pojem human resources může mít negativní konotace a~značit, že lidé jsou zdrojem ve výrobě jako cokoli jiného. \cite[s.~1]{armstrong2014} Samotné spojení human resources se používá ve spojitosti s~personálním oddělením v~rámci organizace. }) je definováno jako komplexní přístup k~zaměstnávání a rozvoji osob. Pojem zahrnuje všechny aspekty toho, jak jsou osoby zaměstnány a řízeny v~rámci organizace. \cite[s.~1]{armstrong2014}

Dalším termínem užívaným v~literatuře je pracovní síla (angl. workforce nebo manpower), resp. její plánování, jedná se o základní proces řízení lidských zdrojů, který je utvářen strategií organizace a jeho cílem je zajistit, aby správný počet lidí se správnými schopnostmi, na správném místě, ve správném čase, za správnou cenu a~ve správném pracovněprávním poměru pomáhal organizaci dosáhnout jejích cílů. \cite{cipd2020_workforce} Mezi kroky tohoto procesu dle \cite{cipd2020_workforce} patří:
\begin{itemize}
	\item analýza aktuální personální situace;
	\item stanovení budoucích potřeb;
	\item identifikace současných nedostatků vzhledem k~plánu do budoucna;
	\item podnikání akcí k~odstranění nedostatků;
	\item monitorování a evaluace akcí.
\end{itemize}

[Kde si v~tom systému stojí rozvrhování???]



\section{Rozhodující faktory při plánování}
% PESTLE analyza

\subsection{Pracovní síly}
V rámci organizace mohou existovat rozdíly mezi jednotlivými pracovníky. Jen část zaměstnanců tak bude pracovat na plný úvazek, jiní mohou pracovat méně hodin, například pouze v nejvytíženějších dnech \cite{lin2015}. V případě těchto zaměstnanců, kteří pracují méně hodin (a ne vždy pravidelně), je třeba zohlednit různé typy pracovněprávních vztahů, čemuž bude věnován prostor dále v~podkapitole \ref{section:legislativa}.
% Individualitu zaměstnanců je třeba zohlednit i z~důvodu rozdílů mezi jejich schopnostmi.
% Dalším hlediskem, které je vhodné brát v~úvahu, je i spokojenost zaměstnanců, jejich časové možnosti a preference.

\subsection{Směny}
\label{sub:smeny}
Jednotlivé organizace se od sebe mohou lišit způsobem, jakým vypisují směny. Existují tak podniky, kde je rozvrh práce pravidelný a stejný pro všechny zaměstnance, ale i ty, kde je provoz dvou- nebo i třísměnný, rozvrh směn je sestavován cyklicky a začátek směn daného zaměstnance se v~jednotlivých dnech liší (nejčastěji směny začínají ráno, odpoledne nebo v~noci). Mezi organizace s~cyklicky sestavovaným rozvhem se řadí nejčastěji provozy, které operují 7~dní v~týdnu, například nemocnice, vězení, policie nebo také restaurace a pobočky řetězců rychlého občerstvení \cite{bechtold1981work}.

Nepravidelný rozvrh směn přitom může mít pro zaměstnance nežádoucí zdravotní účinky, \cite{flo2013shift} uvádí, že zaměstnanci ve vícesměnném provozu trpí nespavostí více než zbytek populace, a to především v~případě, že je mezi směnami kratší než 11hodinová přestávka, [TODO další zdravotní rizika].

\subsection{Úkoly}
% TODO -- může to být sekvence činností při směně nebo fakt, že teď pracuje organizace na něčem?

V~případě, že zaměstnanci jsou při práci vystaveni zdravotním rizikům, je vhodné během směny rozdělit úkoly tak, aby riziko bylo minimalizováno. \cite{wongwien2013ergonomic}

% Provozní úkoly (angl. operational tasks) jsou běžné činnosti, které musí během dne zaměstnanci vykonat \cite{lin2015}. Při jejich plnění se přitom nevytváří nový produkt, ale udržuje se chod organizace (např. se jedná o~administrativní úkoly). Lze je rozdělit na prioritní, plánované a neplánované \cite{miwa2010}.

\subsection{Předpověď poptávky po personálu}
\label{sub:demand}
Obecně řečeno musí zaměstnanci plnit úkoly podle toho, jaké události nastanou. Modelováním poptávky se rozumí proces, jehož výstupem je předpověď přibližného počtu zaměstnanců a jejich očekávaných kompetencí \cite[s.~219]{armstrong2014}, a to na základě očekávaných událostí \cite{ernst2004staff}.
Například prodejce hraček tedy na základě historických dat ví, že nejvíce zákazníků přichází před Vánoci a předpokládá, že tomu tak bude i v~následujícím roce. Na tuto událost bude reagovat tím, že se rozhodne rozšířit otevírací dobu. To celkově zvýší poptávku po personálu.

\subsection{Cena}
Dalším důležitým faktorem při plánování pracovních sil je celková cena lidské práce a otázka její optimalizace při naplnění poptávky.


\section{Legislativní podmínky}
\label{section:legislativa}
Hlavním právním předpisem, který se upravuje problematiku pracovních sil, je v~českém prostředí zákoník práce, který upravuje vztahy vznikající při výkonu závislé práce mezi zaměstnanci a~zaměstnavateli, tedy pracovněprávní vztahy \cite{zakon06_262}.

Zaměstnavatel je dle §~38 zákoníku práce povinen přidělovat zaměstnanci práci podle pracovní smlouvy. Zaměstnanec je pak povinen podle pokynů zaměstnavatele konat osobně práci v~rozvržené týdenní pracovní době.

\subsection{Pracovní doba}
Pracovní dobou se rozumí doba, v~níž je zaměstnanec povinen vykonávat práci pro zaměstnavatele nebo je k tomu na pracovišti připraven. Doba odpočinku není součástí pracovní doby. (§ 78 zákoníku práce)

Stanovená týdenní pracovní doba činí dle §~79 zákoníku práce 40~hodin (mimo výjimky). Pracovní dobu rozvrhuje zaměstnavatel, který určuje začátek a konec směn, a to zpravidla do pětidenního pracovního týdne (§ 81). Délka směny nesmí přesáhnout 12~hodin. U nezletilých zaměstnanců pak nesmí délka směny v~jednotlivém dni překročit 8~hodin a v~jednom týdnu 40~hodin.

\subsection{Rozvrh pracovní doby}
Zaměstnavatel je dle §~84 zákoníku práce povinen vypracovat rozvrh týdenní pracovní doby a seznámit s~ním nebo jeho změnou zaměstnance nejpozději 2~týdny předem (mimo výjimky stanovené zákonem nebo v~případě existence jiné dohody mezi zaměstnancem a zaměstnavatelem). Tento rozvrh musí být v~písemné formě. Pracovní dobu je podle §~90 třeba rozvrhovat s ohledem na nepřetržitý odpočinek mezi koncem jedné směny a začátkem následující (pro zletilé zaměstnance alespoň 11 hodin, pro nezletilé zaměstnance alespoň 12 hodin, v~zákonem stanovených výjimkách lze za určitých podmínek odpočinek zkrátit).

\subsection{Směna, směnný provoz}
TODO

\subsection{Pracovní poměr}
Pracovní poměr mezi zaměstnancem a zaměstnavatelem se podle §~33, odst.~1 zákoníku práce zakládá pracovní smlouvou. Zaměstnavatel má zajišťovat plnění svých úkolů především zaměstnanci v pracovním poměru (§ TODO).

\subsection{Přesčasy}
TODO

\subsection{Zkrácený úvazek}
Podle §~80 zákoníku práce může být mezi zaměstnancem a~zaměstnavatelem sjednána kratší pracovní doba.

\subsection{Dohody o pracích konaných mimo pracovní poměr}
§~77 zákoníku práce stanovuje, že na práci konanou na základě dohod o pracích konaných mimo pracovní poměr se vztahuje úprava pro výkon práce v~pracovním poměru, a to mimo výjimky uvedené v odst.~2 tohoto paragrafu (např. pracovní dobu a~dobu odpočinku nebo dovolenou). Podle §~74 přitom zaměstavatel není zaměstnancům na dohody povinen rozvrhnout pracovní dobu.

\subsubsection{Dohoda o provedení práce}
Podle §~75 zákoníku práce se dohoda o provedení práce uzavírá nejvýše na 300~hodin v~kalendářním roce (doba se u jednoho zaměstnavatele sčítá v~případě, že je dohod uzavřeno více).

\subsubsection{Dohoda o pracovní činnosti}
Podle §~76 zákoníku práce není na základě dohody o pracovní činnosti možné vykonávat práci v~rozsahu překračujícím v~průměru polovinu stanovené týdenní pracovní doby. Toto se posuzuje za celou dobu, po niž je uzavřena, nejdéle však za 52 týdnů. Musí být sjednán rozsah pracovní doby a doba, na niž se sjednává.



\chapter{Analýza systému}

%
% \section{Motivace pro softwarové řešení}
%
% TODO.
% Jak zlepšilo produktivitu adoptování systémů? Jsou zaměstnanci víc spoko? Kolik nám to ušetři peněz?
% People centric system


\section{Uživatelské role}
Jedním z~hlavních cílů celého systému na plánování směn je informovat individuálně zaměstnance o jejich vlastních směnách, proto se jeví vhodnou jeho personalizace, a to na základě uživatelských účtů. Vstupní operací bude přihlášení -- z~toho vyplývá nezbytnost existence role přihlášeného a nepřihlášeného uživatele.

Pro roli přihlášeného uživatele byla identifikována nutnost dalšího rozšíření, a to na základě pracovněprávního vztahu (tj. zaměstnanec v pracovním poměru, zaměstnanec na dohodu o provedení práce a zaměstnanec na dohodu o pracovní činnosti) a věku, resp. toho, zda je zletilý či nikoli. Důvodem pro toto rozšíření je, že každý uživatel bude moci provádět jiné operace dle odlišných pravidel. Uživatelská role ovšem není závislá na tom, zda má zaměstnanec plný nebo zkrácený úvazek, neboť v obou případech platí stejné podmínky, liší se jen týdenní pracovní doba. Stranou zůstává role, kterou má v~rámci tohoto systému zaměstnavatel či jím pověřená osoba (dále jen vedoucí), jež má zodpovědnost za rozvrhování pracovní doby. Ta by měla mít komplexní přehled o rozpisu a v~případě nutnosti i možnost do něj zasáhnout.


\section{Funkční požadavky}
Cílem této podkapitoly je formulovat funkční požadavky, tzn. jaké možnosti by měl systém poskytnout koncovým uživatelům, aniž by bylo specifikováno, jak to bude dělat \cite{lauesen2002}.

\subsubsection{Základní požadavky na systém}
\begin{enumerate}
	\item Jako nepřihlášený uživatel se chci do systému přihlásit uživatelským jménem a heslem.
	\item Jako přihlášený uživatel chci mít možnost se ze systému odhlásit.
	\item Jako uživatel chci mít možnost zobrazit si relevantní nápovědu.
\end{enumerate}

\subsubsection{Požadavky na správu směn}
\begin{enumerate}
	% \item Jako vedoucí chci vypsat časové sloty pro směny mně podřízených zaměstnanců.
	\item Jako vedoucí chci vidět komplexní rozpis směn svých podřízených.
	\item Jako vedoucí chci mít možnost změnit automaticky generovaný rozpis směn.
\end{enumerate}

\subsubsection{Požadavky na splnění legislativy}
Při implementaci systému na plánování směn je žádoucí, aby byl dohled nad dodržováním pracovněprávních předpisů, především těch, které uvádějí konkrétní kvantitativní údaje, automatizován. Proto také byly na základě legislativních podmínek z~podkapitoly~\ref{section:legislativa} stanoveny následující požadavky:

\begin{enumerate}
	\item Systém rozvrhne zaměstnancům v~pracovním poměru pracovní dobu úměrně jejich úvazku, a to vždy s~nejméně dvoutýdenním předstihem.
	\item Systém bude automaticky plánovat ideálně 8hodinové, nejdéle však 12hodinové směny.
	\item Systém při plánování směn nezletilým zaměstnancům rozvrhne v~jednom dni maximálně 8hodinové směny, jejich týdenní pracovní doba nepřekročí 40~hodin.
	\item Systém umožní zaměstnancům na dohodu o provedení práce odpracovat nejvýše 300 hodin v~kalendářním roce.

\end{enumerate}


% \section{Uživatelská práva}
% TODO.
% % Vzhledem k~tomu, že se jedná o interní systém organizace, budou práva nepřihlášeného uživatele omezena jen na velmi malou množinu operací.
%
%
% \section{Existující systémy}
% TODO.
% V~následující kapitole bude provedena analýza vybraných systémů, které lze pro plánování lidských zdrojů použít.

% Konkrétně se jedná o systémy Tamigo\footnote{https://tamigo.com}, Workforce\footnote{https://workforce.com} a When I work\footnote{https://wheniwork.com}, zmínka bude věnována i ERP systémům.


\section{UX analýza}



\chapter{Algoritmy pro rozvrhování směn}
Problém plánování směn je jedním z~rozvrhovacích problémů\footnote{Dalšími takovými problémy jsou např.~sestavování školních rozvrhů nebo jízdních řádů.}, obecně se považuje velmi komplexní, a to i v~případě, že se řeší jen jeho zjednodušená verze (např. se vyhodnocuje jen jedno kritérium a schopnosti zaměstnanců jsou homogenní). V~rámci velkých organizací lze komplexnost tohoto problému snížit například tím, že na sobě nezávislé části mají samostatný rozvrh (např. v~nemocnici je rozvrh ostrahy nezávislý na rozvrhu lékařů na patologii).

Přístupů, jak problém rozvrhování řešit, existuje více, \cite{burke2004state} uvádí tři:
% uvadi jako zdroj nekoho jinyho ale
\begin{enumerate}
	\item tradiční postup, kdy je rozvrh vytvořen ručně;
	\item cyklické rozvrhování, které se stěží přizpůsobuje požadavkům zaměstnanců;
	\item rozvrhování s~pomocí počítače, které zrychluje proces i umožňuje plnění požadavků.
\end{enumerate}

Problém je považován za NP-úplný, jeho výzkum se zaměřuje především na vývoj efektivních přibližných metod řešení \cite{adamuthe2012tabu}. Rozvrhovací problém lze algoritmicky řešit více způsoby -- jedním z~nich je jeho formulace, aby odpovídal problému již známému a řešitelnému standardními metodami\footnote{Může se jednat např. o~lineární programování, které nalezne optimální řešení, ovšem jen jedná-li se o~méně komplexní problémy (např. je obtížné formulovat všechny zbytné podmínky pouze jako lineární funkce). \cite{blochliger2004modeling}}, dalším je heuristika, která umožňuje řešení komplexnějších problémů, avšak ne vždy zaručuje nalezení optimálního řešení. \cite{blochliger2004modeling}

Teoretických problémů, které byly popsány v literatuře a které se týkají rozvrhování lidských zdrojů, existuje více druhů dle podmínek a prostředí, jehož se bezprostředně týkají, jedním z~nich je tzv.~problém rozvrhování zdravotních sester (angl. nurse rostering problem nebo nurse scheduling problem).



% Operační výzkum

\section{Definice problému}
Máme k~dispozici $N$~zaměstnanců, které je třeba rozdělit do $S$ směn v~$D$~pracovních dnech, a~to na základě sady podmínek (dle charakteru provozu).

\section{Proces rozvrhování směn}
Dle \cite{ernst2004staff} lze proces rozvrhování rozdělit na několik částí, které mohou, ale nemusí být postupnými kroky tohoto procesu. Konkrétně se jedná o:
\begin{enumerate}
	\item předpovídání poptávky po personálu;
	\item rozvrhování volných dnů;
	\item rozvrhování směn;
	\item rozvrhování prací;
	\item rozdělení úkolů;
	\item přidělení osob.
\end{enumerate}

\section{Podmínky}
\label{sec:constraints}
Rozvrhování směn závisí na takovém množství podmínek, že je obvykle není možné splnit všechny, proto se také často musí před samotným řešením problému rozdělit na nezbytné~(hard, značeno $H_i$, musí být splněny vždy) a~zbytné~(soft, značeno $S_i$, jejich splnění je žádoucí). \cite{todorovic2012bee} Pro každou podmínku je třeba stanovit její váhu, nezbytné podmínky mají například váhu v~řádu vyšších stovek (mezi 500 a 1000) a zbytné podmínky mají váhu řádově menší (mezi 50 a 150). \cite{buyukozkan2014applicability}

Podmínky se mohou týkat například sekvence činností (např. po náročné chirurgické operaci může následovat pouze administrativní činnost nebo po směně nemůže následovat 12 hodin další směna), počtu (např. zaměstnanec pracuje týdně~$ 40~\mbox{h} \pm 8~\mbox{h}$), nekompatibilních stavů (např. Alice nechce pracovat s~Bobem) nebo nezbytných činností (např. je v~daném dni naplánována náročná operace srdce). Lze je rozdělit na globální, jejichž posouzení vyžaduje celkovou znalost řešení, a lokální, na jejichž verifikaci stačí pohled na vlastní podmnožinu řešení. Toto rozdělení je naznačeno na obr.~\ref{fig:constraints}, nezbytné činnosti však nelze rozdělit jednoznačně. \cite{blochliger2004modeling}

\begin{figure}
	\input{img/constraints.pdf_tex}
	\caption{Rozdělení podmínek dle jejich rozsahu}
	\label{fig:constraints}
\end{figure}



\section{Vstupy}
Organizace má množinu zaměstnanců $M$, z~nichž každý zaměstnanec $m \in M$ má určenou množinu činností $Q_m$, které je kvalifikován vykonávat; podmnožině zaměstnanců $E \subseteq M$, $E = \{ 1, 2,~\ldots, e \}$ v~daném období může být rozvrhnuta směna. Každý zaměstnanec přitom v jednom čase může být pouze na jednom místě.

\begin{figure}[h]
	\input{img/problem-definition.pdf_tex}
	\caption{Ilustrace problému}
	\label{fig:definition}
\end{figure}

Rozvrh se vytváří na období $T = \{1, 2,~\ldots, t \}$ (typicky je toto období rozděleno na dny), v~každém dni jsou vypsány směny $S = \{ 1, 2,~\ldots, s\}$, které jsou charakterizovány např. časovým intervalem nebo požadovanou kvalifikací.

Byla použita některá z~metodik na předpovídání poptávky po personálu (viz podkapitolu \ref{sub:demand}), výsledkem je tabulka (viz např. \ref{tab:demand}), v~buňkách je potřebný počet zaměstnanců v~daném čase (poptávku může určit také rozmezí počtu zaměstnanců  nebo to může být relativní veličina).

\begin{table}[h]
	\input{tables/demand-tabular.tex}
	\caption{Příklad týdenní poptávky}
	\label{tab:demand}
\end{table}

Dalším vstupem je tabulka s~nezbytnými (tab.~\ref{tab:hard}) a zbytnými (tab.~\ref{tab:soft}) požadavky.

\begin{table}[h]
	\begin{tabular}{|@{\makebox[3em][c]{${H}_{\rownumber}$}} |l|c|}
	\hline
	\multicolumn{1}{|@{\makebox[3em][c]{ID}} | l |}{ Požadavek } & Váha\\
	\hline
	Všechny směny musí být zaplněny. & 1000 \\
	{Každý pracovník může pracovat nejvýše jednu směnu denně.} & 1000 \\
	\hline
	\end{tabular}

	\caption{Příklad nezbytných podmínek}
	\label{tab:hard}

\end{table}

\begin{table}[h]
	\begin{tabular}{|@{\makebox[3em][c]{${S}_{\rownumber}$}} |p{0.8\linewidth}|c|}
	\hline
	\multicolumn{1}{|@{\makebox[3em][c]{ID}} | l |}{ Požadavek } & Váha\\
	\hline
	{Jsou dodrženy požadavky zaměstnanců na volno.} & 100 \\
	{Dvojice zaměstnanců nechce pracovat společně.} & 50 \\
	\multicolumn{1}{|@{\makebox[3em][c]{$\vdots$ }} | c |}{ $\vdots$  } & $\vdots$ \\
	\hline
	\end{tabular}

	\caption{Příklad zbytných podmínek}
	\label{tab:soft}
\end{table}

\section{Interpretace řešení}
Řešení tohoto problému lze interpretovat jako trojrozměrnou binární matici $\boldsymbol{x}$, jejímiž parametry jsou zaměstnanec ($\forall i \in E$), den ($\forall j \in T$) a vypsaná směna $\forall k \in S$, jednotlivé hodnoty $x_{ijk}$ jsou určeny funkcí \ref{eq:3dmatrix}. \cite{vaclavik2016roster}

\begin{equation}
	\label{eq:3dmatrix}
	x_{ijk} =
	\begin{cases}
		1, & \mbox{zaměstnanci $i$ je v den $j$ přiřazena směna $k$} \\
		0, & \mbox{jinak}\\
	\end{cases}
\end{equation}

\begin{table}
	\input{tables/interpretation-example.tex}
	\caption{}
	\label{}

\end{table}

\section{Cílová funkce}
\label{sec:objective}
Cílová nebo také účelová funkce (angl. objective function) je taková funkce, jejíž hodnotu je cílem optimalizovat. V~případě rozvrhování se může jednat o funkce vyjadřující rovnoměrnost obsazení směn, cenu, porušené zbytné podmínky \cite{blochliger2004modeling} či počet kontaktů mezi zaměstnanci (důležitý např. v~případě epidemie) \cite{zucchi2020personnel}.

Například v~případě optimalizace na základě vážených podmínek dle podkapitoly \ref{sec:constraints} ji lze formulovat jako
\begin{equation}
	f(\boldsymbol{x}) = \sum_{s = 1}^n c_s \cdot g_s(\boldsymbol{x}),
\end{equation}
kde $\boldsymbol{x}$ je dané řešení, $n$ je počet zbytných podmínek, $c_s$ je váha~$s$-té~podmínky (také lze tento údaj chápat jako sankci za její porušení) a $g_s(\boldsymbol{x})$ je počet porušení $s$-té~podmínky v~řešení~$\boldsymbol{x}$. \cite{awadallah2015hybrid}

\section{Lineární programování}
Úlohou lineárního programování\footnote{Termín programování zde souvisí se slovem program ve významu plán nebo rozvrh, nikoli s~počítačovými programy. } je nalézt vektor $\boldsymbol{x}^{\ast} \in \mathbb{R}^n$ optimalizující hodnotu cílové funkce mezi všemi vektory, které splňují danou soustavu lineárních rovnic a nerovnic (kterým se zpravidla říká omezující podmínky nebo omezení). \cite{matousek2006linearni}

\subsection{Celočíselné lineární programování}

Celočíselné lineární programování je pro rozvrhování směn vhodné v jednoduchých případech, např. když jsou pro všechny zaměstnance stanoveny stejné počty po sobě jdoucích pracovních dnů a volna nebo, jsou-li rozvrhovány směny v rámci dne, je den rozdělen na několik disjunktních úseků, navíc je stanoven počet po sobě následujích úseků, které tvoří jednu směnu. Pro každý časový úsek (den či část dne) je navíc stanoven minimální počet personálu. \cite{satheeshkumar2014linear} Cílem je optimalizovat počet zaměstnanců tak, aby byla minimální poptávka naplněna.

V~případě týdne, v~němž mají zaměstnanci 5 po sobě jdoucích pracovních dnů a 2 po sobě jdoucí dny volna, je rozdělení ilustrováno na obr. \ref{fig:linear}. Příklad minimální poptávky po personálu je uveden v~tab.~\ref{tab:linear}.

\begin{figure}[h]
	\input{img/linear-programming-days.pdf_tex}
	\caption{Rozdělení pěti po sobě následujích dní}
	\label{fig:linear}
\end{figure}

\begin{table}[h]
	\input{tables/linear-programming-demand.tex}
	\caption{Příklad minimální poptávky po personálu}
	\label{tab:linear}
\end{table}

Cílem je na základě těchto údajů zjistit, jaký počet zaměstnanců $z$ je třeba povolat, aby $z = \sum_{i=1}^{7} x_i$ (jedná se rovněž o cílovou funkci, jež byla zmíněna v~podkapitole \ref{sec:objective}), kde $x_i$ je počet zaměstnanců, jejichž pracovní týden začíná v $i$-tém dni, bylo co nejmenší.

Na základě~tab.~\ref{tab:linear} a~s~pomocí obr.~\ref{fig:linear} lze přitom sestavit několik podmínek, s jejichž pomocí lze $z$ vypočítat\footnote{Například použítím vhodného software.}:
\begin{center}
	\begin{tabular}{rl}
		$\forall x_i: x_i \geq 0$; & $x_1 + x_4 + x_5 + x_6 + x_7 \geq 200$; \\
		$x_1 + x_2 + x_5 + x_6 + x_7 \geq 150$; & $x_1 + x_2 + x_3 + x_6 + x_7 \geq 250$;\\
		$x_1 + x_2 + x_3 + x_4 + x_7 \geq 90$; & $x_1 + x_2 + x_3 + x_4 + x_5 \geq 160$; \\
		$x_2 + x_3 + x_4 + x_5 + x_6 \geq 300$; & $x_3 + x_4 + x_5 + x_6 + x_7 \geq 100$.
	\end{tabular}
\end{center}

\subsection{Omezení algoritmu}

Lineární programování může přinést optimální výsledky, avšak je vhodné pouze v~případech, kdy lze pro podmínky sestavit soustavu lineárních rovnic a nerovnic. Časová složitost v~závislosti na počtu zaměstnanců je exponenciální. \cite{chen2016comparison}


\newpage
\section{Optimalizace včelím rojem}
Optimalizace včelím rojem (angl. bee colony optimization) je evoluční algoritmus, v~němž každá ze \uv{včel} hledá řešení. Tato heuristická metoda se~skládá ze dvou fází, dopředné~a~zpětné, které se opakují, dokud není splněna výstupní podmínka.

\subsection{Umělý včelí roj}
Umělý včelí roj se v~původní verzi algoritmu skládá ze tří skupin včel (dělnice, průzkumnice, vyčkávající včely), které vyhledávají zdroje potravy (které symbolizují možné řešení problému), množství nektaru symbolizuje cílovou funkci. \cite{anuar2016modified}

\subsection{Inicializační fáze}
Dojde k inicializaci zdrojů potravy $\boldsymbol{x}_m$ pro $m \in \{1,~\ldots~, SN\}$ včelami průzkumnicemi, každý vektor $\boldsymbol{x}_m = (x_{mi},~i \in \{1,~\ldots, n \})$ je řešením problému o $n$ proměnných, které je třeba optimalizovat, v~inicializační fázi může $x_{mi}$ odpovídat definici~\ref{eq:initialization}, kde $l_i$ (resp. $u_i$) je dolní (resp. horní) mez parametu $x_{mi}$. \cite{karaboga2010artificial}

\begin{equation}
	\label{eq:initialization}
	x_{mi} = l_i + \mbox{rand}(0,1) \cdot (u_i - l_i),
\end{equation}


% Úkolem včel dělnic je vyletět ke zdroji potravy a vrátit se do úlu (\textit{dopředná fáze}), kde sdílejí informace o svých zdrojích potravy s~vyčkávajícími včelami (způsob sdílení se nazývá tancem). Ze včel dělnic, jejichž řešení bylo vyčerpáno, se stávají průzkumnice, které mohou náhodně hledat nové zdroje potravy (\textit{zpětná fáze}).

\subsubsection{Dopředná fáze}
V~první iteraci se vytvoří $SN$ náhodných řešení, $SN$ odpovídá počtu včel dělnic a průzkumnic. V~každé další iteraci dělnice upravují své řešení náhodně tak, že pokud je lepší než předchozí, zapamatují si ho \cite{anuar2016modified}. Vyhledávací proces pokračuje, dokud není dosaženo maximálního počtu cyklů nebo přijatelné hodnoty \cite{banharnsakun2011best}.

\subsubsection{Zpětná fáze}
Ve zpětné fázi si včely na základě hodnoty cílové funkce sdílejí informace o~kvalitě svých řešení. Pak se každá včela náhodně (pravděpodobnost výběru $i$-tého~řešení $p_i$ bude distribuována dle kvality řešení -- cílové funkce $f(\boldsymbol{x}_i)$, jak je naznačeno v~rovnici \ref{eq:beeprobability}) rozhodne, zda své řešení bude prosazovat či nikoli. Pokud nepokračuje, vybere náhodně to řešení, ke kterému se přikloní, a to si zapamatuje. \cite{teodorovic2009bee}

\begin{equation}
	\label{eq:beeprobability}
	p_i = \frac{f(\boldsymbol{x}_i)}{\sum_{n = 1}^{SN} f(\boldsymbol{x}_n)}
\end{equation}

\newpage
\subsection{Algoritmus}
Máme kolonii včel o~velikosti~$B$, která může udělat $NF$ iterací. V jedné iteraci může včela udělat $NC$ konstrukčních kroků, v~každém jednotlivém konstrukčním kroku musí rozřadit $NS$~směn mezi $NS$~různých zaměstnanců. V~každém konstrukčním kroku se evaluují všechny kombinace, aby se nalezlo nejlepší řešení. \cite{khader2013artificial}

Zjednodušený průběh optimalizace včelím rojem je zobrazen na vývojovém diagramu~\ref{fig:beeflow}.
\begin{figure}[h]
	\input{img/img_bees_flow.pdf_tex}
	\caption{Zjednodušený průběh algoritmu}
	\label{fig:beeflow}
\end{figure}

V~textové podobě pak dle \cite{rajeswari2017directed} algoritmus vypadá následovně:
\begin{lstlisting}
Vytvoř včely.
Opakuj
	Iteruj přes všechny včely.
		Nastav čítač na 1.
		Opakuj
			Vyhodnoť všechny možné konstrukční kroky.
			Vyber náhodně jedno řešení.
			Zvyš čítač o 1.
		dokud je hodnota čítače menší než celkový počet konstrukčních kroků.
	Vrať se do úlu.
	Spočítej cílovou funkci pro každou včelu; seřaď řešení dle hodnoty.
	Včela s nejlepším výsledkem se stane vedoucím.
	Každá včela se náhodně stane vůdcem nebo následovníkem. Následovník zahodí své řešení.
	Každý následovník se přidá k některému z vůdců a převezme jeho řešení.
dokud nebyla splněna výstupní podmínka.
Vyhodnoť a vyber nejlepší řešení.
Vrať nejlepší řešení.
\end{lstlisting}

\subsection{Složitost algoritmu}

Původní algoritmus optimalizace včelím rojem má složitost $O(mnd)$, kde $n$ je celkový počet včel, $m$ je maximální pošet iterací a $d$ je dimenze řešení. \cite{banharnsakun2011best}


% https://www.sciencedirect.com/science/article/pii/S1319157816300039

% PEAST algorithm = http://www.lnse.org/papers/51-CA009.pdf
% Employee scheduling in service industries with flexible employee availability and demand https://www.sciencedirect.com/science/article/pii/S0305048316000475#bib1

% \newpage
% \subsection{Rozvrhování prací}
% % TODO
% [TODO: Nechat to tu?]
%
% Mezi jednotlivými pracemi (úkoly) může být vztah precedence, tzn., že nezbytnou podmínkou pro začátek některé z~prací je ukončení jiné práce (jedná se tedy o~uspořádané dvojice). Posloupnost prací lze zobrazit do orientovaného grafu \cite{mohring2004scheduling}, který lze zobrazit dvěma způsoby, buďto jsou práce vrcholy (obr.~\ref{fig:node}), nebo jsou práce hrany (obr.~\ref{fig:arc}), v~tom případě jsou vrcholy milníky. \cite[s.~51--57]{pinedo2005planning}
%
% % % TODO obrázky
% \begin{figure}[h]
% \def\svgwidth{\columnwidth}
% \input{img/job_on_node.pdf_tex}
% \caption{Práce jako vrchol}
% \label{fig:node}
% \end{figure}
%
% \begin{figure}[h]
% \def\svgwidth{\columnwidth}
% \input{img/job_on_arc.pdf_tex}
% \caption{Práce jako hrana}
% \label{fig:arc}
% \end{figure}


\section{Cyklické rozdělení}
Hlavní myšlenkou metody cyklického rozdělení je, že směny lze mezi zaměstnance rozdělit na základě opakujících se schémat. Skládá se ze tří fází: dekompozice, konstrukce a následné zpracování. \cite{brucker2005decomposition} Hlavním problémem zde je nalezení schémat, podproblémem jejich sestavení do rozvrhu. \cite{becker2020decomposition}

\subsection{Dekompozice}
Zaměstnanci jsou rozděleni do několika skupin. Nejprve se sestaví schéma pro směny, jejichž zaplnění je nejtěžší (tzn., že způsobu jejich zaplnění se týkají největší sankce). Na základě toho se vytvoří schéma cyklicky se opakujících bloků (např. schéma pro jeden týden). Vytváření schématu pokračuje i pro další druhy směn a cílem je, aby se co nejvíce směn rozdělovalo cyklicky dle bloků a co nejméně jich bylo třeba rozdělit dodatečně.

\subsection{Konstrukce}
Směny, které nejsou rozděleny dle cyklického schématu v~první fázi, se rozdělí v~druhé fázi. Součástí může být i přesunutí některých už přiřazených směn.

\subsection{Následné zpracování}
V poslední fázi se provede opětovný průzkum prostoru. Sousedící směny se mohou přeházet, pokud se zdá, že toto řešení bude lepší. Výsledkem nemusí být cyklicky se opakující rozvrh.

\newpage
\section{Návrh vlastního algoritmu}

\subsection{Požadavky na algoritmus}
Na základě potřeb tohoto systému byly stanoveny požadavky na algoritmus uvedené v~tab.~\ref{tab:algorithm_req}. Tyto požadavky se liší od standardních požadavků na výše uvedené algoritmy pro rozvrhování zdravotních sester, neboť toto řešení předpokládá, že volné směny budou dále rozvrhnuty jiným způsobem. Zároveň je zde cílem netvořit zbytečné překážky v~zaměstnání, tedy rozvrhnout práci všem.

\begin{table}[h!]
	\input{tables/algorithm_req.tex}
	\caption{Požadavky na algoritmus}
	\label{tab:algorithm_req}
\end{table}

\subsection{Průběh}
Rozdělování směn bude probíhat po menších skupinách, prioritně se rozdělí směny pro ty zaměstnance, kteří mají nejspecifičtější požadavky (a je tedy nejtěžší je přidělit). V~počátečním stavu je rozvrh prázdný, zaměstnancům nejsou přiděleny žádné směny. Rozvrh se sestavuje po dnech, v~každém dni se rozdělí směny mezi zaměstnance z~aktuální podskupiny (viz obr. \ref{fig:dayflow}).

\begin{figure}[h]
	\input{img/flowchart-algo-day.pdf_tex}
	\caption{Zjednodušený průběh algoritmu}
	\label{fig:dayflow}
\end{figure}

Rozložení pravděpodobnosti při náhodném výběru vhodné směny\footnote{Směna vhodná pro daného zaměstnance je taková, která splňuje nezbytné požadavky -- neuvažuje se tedy např. ta směna, která začíná méně než 12~h po konci minulé směny.} nebo volna může záviset na několika faktorech, např.:
\begin{enumerate}
	\item prioritu má zaplnění rozvrhu ve všech časech, kdy jsou vypisovány směny;
	\item preferováno je zaplnění relativně ku poptávce;
	\item preferovány jsou celé 8hodinové směny\footnote{Je-li to však třeba, lze je i rozdělit.};
	\item preferováno je více dnů volna pohromadě.
\end{enumerate}


\subsection{Příklad}
Předpokládá se, že organizace má $N = 10$ zaměstnanců, kteří mají různě velký úvazek a mohou pracovat v~různé dny (tab. \ref{tab:algorithm_employees}). Všichni mají stejné schopnosti a jejich specializace není určena. Navíc byl sestaven i~model vytíženosti (je relativní; vyšší číslo značí, že v~daný čas je vyšší poptávka po zaměstnancích), který je v~tab. \ref{tab:algorithm_demand}.

\begin{table}[h]
	\input{tables/algorithm_employees.tex}
	\caption{Příklad struktury zaměstnanců}
	\label{tab:algorithm_employees}
\end{table}

\begin{table}[h]
	\input{tables/algorithm_demand.tex}
	\caption{Příklad poptávky po zaměstnancích}
	\label{tab:algorithm_demand}
\end{table}

\subsubsection{Rozdělení zaměstnanců}
V prvním kroku jsou vybráni zaměstnanci, kteří mají nejspecifičtější potřeby, tedy 9 a 10, kteří mohou pracovat nejméně dnů v~týdnu.

\paragraph{Pondělí až čtvrtek} Směny mezi tyto zaměstnance rozdělit nelze, neboť z~tab.~\ref{tab:algorithm_employees} vyplývá, že v~tyto dny není ani jeden z~těchto zaměstnanců k~dispozici.

\paragraph{Pátek až neděle} Ve~zbývajících dnech je naopak nezbytné rozvrh pro tyto zaměstnance sestavit\footnote{Předpokládá se, že za týden musí odpracovat při svém úvazku právě 20~hodin. Každý den odpracují nejvýše 8~hodin, tzn., že 20~hodin lze rozdělit mezi 3 dny.}.

\subparagraph{Pátek} Náhodná funkce přidělí zaměstnance~9 na ranní směnu (zatím v~daném čase není na pracovišti žádný zaměstnanec, navíc jde o vytížený čas, pravděpodobnost tohoto kroku je tedy nejvyšší). Zaměstnanci~10 je přidělena odpolední směna.

\subparagraph{Sobota} Zaměstnanci~9 může být přidělena ranní nebo odpolední směna, z~této množiny je mu náhodně vybrána ranní. Zaměstnanec~10 má v~sobotu specifičtější požadavky -- musí mezi směnami mít nejméně 12hodinovou přestávku, a tu by neměl, kdyby pracoval celou ranní směnu, v~úvahu tedy připadá pouze odpolední směna (celá či její část), případně část ranní směny (stále je třeba rozdělit 12~hodin mezi 2 dny). Je mu přidělena např. celá odpolední.

\subparagraph{Neděle} Zaměstnanec~9 opět může pracovat ráno i odpoledne, náhodně se vybere odpoledne, směnu je však nezbytné zkrátit (zbývá rozdělit 4~hodiny, směna má běžně 8 hodin). Je mu tedy přidělena úvodní část odpolední směny. Zaměstnanci~10 může být přidělena jen část odpolední směny, vzhledem k~tomu, že je část odpolední směny neobsazená, je mu přiřazen její konec.

Po prvním rozdělení je tak v~daném čase počet zaměstnanců dle tab.~\ref{tab:firstiteration}, rozvrh prvních dvou zaměstnanců je v~tab.~\ref{tab:firstemployees}.

\begin{table}[h]
	\input{tables/example_first_iteration.tex}
	\caption{Počet zaměstnanců v~daném čase po prvním rozdělení.}
	\label{tab:firstiteration}

\end{table}

\begin{table}[h]
	\input{tables/example_first_iteration_employees.tex}
	\caption{Směny prvních zaměstnanců po prvním rozdělení}
	\label{tab:firstemployees}

\end{table}

Postup při rozdělení dalších zaměstnanců je analogický, validním výsledkem může být například tab.~\ref{tab:exemployees}.

\begin{table}[h]
	\input{tables/example_employees.tex}
	\caption{Rozvrh směn na týden}
	\label{tab:exemployees}

\end{table}

\begin{table}[h]
	\input{tables/example_count.tex}
	\caption{Počet zaměstnanců v~daném čase}
	\label{tab:excount}

\end{table}

% \chapter{Backend}
%
%
% \section{Autentizace}
% Autentizace je realizována pomocí knihovny Devise Token Auth\footnote{\url{https://github.com/lynndylanhurley/devise_token_auth}}, server v~případě úspěšné autentizace odešle klientovi v~hlavičce token typu bearer (tzn. nositel, \uv{umožni nositeli tohoto tokenu přístup} \cite{swagger2020bearer}) s~platností na dva týdny. Akce bude uživateli umožněna v~případě, že~na server odešle v~HTTP~hlavičce spolu se svým požadavkem i~údaje o~tokenu.
%
% Hesla uživatelů se v~databázi uchovávají v~hashované podobě.
%
%
%
% \chapter{Mobilní aplikace}
%
%
% \section{Design}
%
%
%
% \chapter{Závěr}

\printbibliography[title={Seznam použité literatury}]

\end{document}
